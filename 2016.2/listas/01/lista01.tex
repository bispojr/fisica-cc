\documentclass[12pt,a4paper,oneside]{article}

\usepackage[utf8]{inputenc}
\usepackage[portuguese]{babel}
\usepackage[T1]{fontenc}
\usepackage{amsmath}
\usepackage{amsfonts}
\usepackage{amssymb}
\usepackage{graphicx}

\author{\\Universidade Federal de Goiás - UFG (Regional Jataí) \\Bacharelado em Ciência da Computação \\Física para Ciência da Computação \\Prof. Esdras Lins Bispo Jr.}

\title{
	{\sc \huge Lista de Exercícios 1} 
	\\{\tt Versão 1.0}
}

\begin{document}

\maketitle

\begin{enumerate}

\section{Conceitos}

	\item Apresente ao menos dois argumentos que justifique a relevância da Física para a Ciência da Computação.
	
\section{Programação}
	
	\item Em JavaScript, crie um objeto	{\tt Carro} que tenha as propriedades (i) {\tt placa}, (ii) {\tt cor}, (iii) {\tt velocidadeMaxima}, e (iv) {\tt relatorio}. A {\tt placa} e a {\tt cor} são cadeias; a {\tt velocidadeMaxima} é um número fracionário; e {\tt relatorio} é uma função que exibe, via {\tt console.log}, todas as demais propriedades de {\tt Carro}. Escolhas valores para as propriedades ao seu gosto. 
	
	\item Em JavaScript, crie um objeto {\tt Calculadora} que tenha as propriedades de {\tt somar}, {\tt dividir}, {\tt subtrair} e {\tt multiplicar}. Todas estas propriedades são operações binárias, recebem valores inteiros e retornam valores inteiros. Se, para as entradas fornecidas, não for possível gerar um valor de retorno válido, então exiba, via {\tt console.log}, o motivo do não retorno do valor.
		
\end{enumerate}

\section{Referências}

\begin{itemize}
	\item HALLIDAY, D.; RESNICK, R.; KRANE, K. Física v1, 4ª Edição, LTC, Rio de Janeiro, 2003.

	\item RAMTAL, D.; DOBRE, A. Physics for JavaScript Games, Animation, and Simulations with HTML5 Canvas, Apress, 2014.
\end{itemize}

\end{document}