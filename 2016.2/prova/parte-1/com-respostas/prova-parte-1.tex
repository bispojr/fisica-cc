\documentclass[12pt,a4paper,oneside]{article}

\usepackage[utf8]{inputenc}
\usepackage[portuguese]{babel}
\usepackage[T1]{fontenc}
\usepackage{amsmath}
\usepackage{amsfonts}
\usepackage{amssymb}
\usepackage{graphicx}
\usepackage{pgfplots}

\usepackage{xcolor}
% Definindo novas cores
\definecolor{verde}{rgb}{0.25,0.5,0.35}
\definecolor{jpurple}{rgb}{0.5,0,0.35}
% Configurando layout para mostrar codigos Java
\usepackage{listings}
\definecolor{mygreen}{rgb}{0,0.6,0}
\definecolor{mygray}{rgb}{0.5,0.5,0.5}
\definecolor{mymauve}{rgb}{0.58,0,0.82}

\lstdefinelanguage{JavaScript}{
	keywords={typeof, new, true, false, catch, function, return, null, catch, switch, var, if, in, while, do, else, case, break},
	keywordstyle=\color{blue}\bfseries,
	ndkeywords={class, export, boolean, throw, implements, import, this},
	ndkeywordstyle=\color{darkgray}\bfseries,
	identifierstyle=\color{black},
	sensitive=false,
	comment=[l]{//},
	morecomment=[s]{/*}{*/},
	commentstyle=\color{purple}\ttfamily,
	stringstyle=\color{red}\ttfamily,
	morestring=[b]',
	morestring=[b]",
}

\lstset{ %
	backgroundcolor=\color{white},   % choose the background color; you must add \usepackage{color} or \usepackage{xcolor}
	basicstyle=\small,        % the size of the fonts that are used for the code
	breakatwhitespace=false,         % sets if automatic breaks should only happen at whitespace
	breaklines=true,                 % sets automatic line breaking
	captionpos=b,                    % sets the caption-position to bottom
	commentstyle=\color{mygreen},    % comment style
	deletekeywords={...},            % if you want to delete keywords from the given language
	escapeinside={\%*}{*)},          % if you want to add LaTeX within your code
	extendedchars=true,              % lets you use non-ASCII characters; for 8-bits encodings only, does not work with UTF-8
	frame=single,	                   % adds a frame around the code
	keepspaces=true,                 % keeps spaces in text, useful for keeping indentation of code (possibly needs columns=flexible)
	keywordstyle=\color{blue},       % keyword style
	language=HTML,                 % the language of the code
	otherkeywords={*,...},           % if you want to add more keywords to the set
	numbers=left,                    % where to put the line-numbers; possible values are (none, left, right)
	numbersep=5pt,                   % how far the line-numbers are from the code
	numberstyle=\tiny\color{mygray}, % the style that is used for the line-numbers
	rulecolor=\color{black},         % if not set, the frame-color may be changed on line-breaks within not-black text (e.g. comments (green here))
	showspaces=false,                % show spaces everywhere adding particular underscores; it overrides 'showstringspaces'
	showstringspaces=false,          % underline spaces within strings only
	showtabs=false,                  % show tabs within strings adding particular underscores
	stepnumber=1,                    % the step between two line-numbers. If it's 1, each line will be numbered
	stringstyle=\color{mymauve},     % string literal style
	tabsize=2,	                   % sets default tabsize to 2 spaces
	title=\lstname,                   % show the filename of files included with \lstinputlisting; also try caption instead of title
	moredelim=**[is][\color{purple}]{@}{@},
}

\author{\\Universidade Federal de Goiás (UFG) - Regional Jataí\\Bacharelado em Ciência da Computação \\Física para Ciência da Computação \\Esdras Lins Bispo Jr.}

\title{\sc \huge Prova (Parte 1)}

\date{28 de março de 2017}

\begin{document}

\maketitle

{\bf ORIENTAÇÕES PARA A RESOLUÇÃO}

\footnotesize

\begin{itemize}
	\item A avaliação é individual, sem consulta;
	\item A pontuação máxima desta avaliação é 10,0 (dez) pontos, sendo uma das 05 (cinco) componentes que formarão a média final da disciplina: dois testes, duas provas e exercícios-bônus;
	\item A média final ($MF$) será calculada assim como se segue
	\begin{eqnarray}
		MF & = & MIN(10, S) \nonumber \\
		S & = & (\sum_{i=1}^{4} 0,2.T_i ) + 0,2.P  + EB \nonumber
	\end{eqnarray}
	em que 
	\begin{itemize}
		\item $S$ é o somatório da pontuação de todas as avaliações,
		\item $T_i$ é a pontuação obtida no teste $i$,
		\item $P$ é a pontuação obtida na prova, e
		\item $EB$ é a pontuação total dos exercícios-bônus.
	\end{itemize}
	\item O conteúdo exigido compreende os seguintes pontos apresentados no Plano de Ensino da disciplina: (1) Fundamentos Matemáticos, (2) Medidas Físicas e Vetores, e (3) Movimentos.
\end{itemize}


\begin{center}
	\fbox{\large Nome: \hspace{10cm}}
	\fbox{\large Assinatura: \hspace{9cm}}
\end{center}

\newpage

\normalsize

\begin{enumerate}
	
	\section*{Substitutiva do Teste 01}

	\item (5,0 pt) {\bf (Halliday 1.20)} O recorde para a maior garrafa de vidro foi estabelecido em 1992 por uma equipe de Millville, Nova Jersey, que criou uma garrafa com um volume de 193 galões americanos (admita que 1 galão americano seja 3785,41 cm$^3$, e a massa específica da água seja 1 g/cm$^3$).
		\begin{enumerate}
			\item Qual é a diferença entre esse volume e 1,0 milhão de centímetros cúbicos?\\
			{ \color{blue}
				{\bf Resposta:} \\
				(i) esse volume em cm$^3$:
				\begin{center}
					$193 \times 3.785,41$ = 730.584,13 cm$^3$
				\end{center}
				(ii) a diferença entre esse volume e 1,0 milhão de centímetros cúbicos:
				\begin{center}
					730.584,13 - 1.000.000 = -269.415,87 cm$^3$
				\end{center}
				
			}
			\item Se a garrafa fosse enchida com água a uma vazão de 1,8 g/min, em quanto tempo estaria cheia?\\
			{ \color{blue}
				{\bf Resposta:} \\
				(i) massa em grama de água da garrafa cheia:
				\begin{center}
					730.584,13 cm$^3 \times 1$ g/cm$^3$ = 730.584,13 g
				\end{center}
				(ii) tempo em minutos para o enchimento da garrafa:
				\begin{center}
					(730.584,13 g) / (1,8 g/min) $\cong$ 405.880,07 min\\
					405.880,07 min $\cong$ 281 d, 20 h, 40 min e 4 s 
				\end{center}
				
			} 
		\end{enumerate}
	
	\newpage
	
	\item (5,0 pt) Em JavaScript, crie um protótipo de objeto {\tt Calculadora} que tenha as propriedades de {\tt somar} e {\tt dividir}. Todas estas propriedades são operações binárias, recebem valores inteiros e retornam valores inteiros. Se, para as entradas fornecidas, não for possível gerar um valor de retorno válido, então exiba, via {\tt console.log}, o motivo do não retorno do valor.
	
	{\color{blue} Resposta: }
	
	\begin{lstlisting}[language=JavaScript]
	function Calculadora(){
		this.somar = function(a, b){
			var sum = a + b;
			if(Math.round(sum) == sum){
				return sum;
			}else{
				console.log(
					"Pelo menos uma das parcelas 
					nao eh um inteiro.");
			}
		};
		this.dividir = function(a, b){
			if(b == 0){
				console.log(
				   "Nao eh permitida realizar 
				   a divisao por zero. Informe um outro valor."
				);
			}else{
				if(Math.round(a) == a){
					var div = a / b;
					if(Math.round(div) == div){
						return div;
					}
					else{
						console.log(
							"O resultado da divisao 
							nao eh inteira."
						);
					}
				}
			}
		};
	}\end{lstlisting}
	
	\newpage
	
	\section*{Substitutiva do Teste 02}
	
	\item (5,0 pt) {\bf (Halliday 2.14)} A função posição $x(t)$ de uma partícula que está se movendo ao longo do eixo $x$ é $x = 4,0 - 6,0t^2$, com $x$ em metros e $t$ em segundos. \label{q:2-14}
	\begin{enumerate}
		\item (0,5 pt) Em que instante e\\
		{ \color{blue}
			{\bf Resposta:} \\
			(i) obter v(t) a partir de x(t):
			\begin{center}
				v(t) = x'(t) $\therefore$\\
				v(t) = -12t
			\end{center}
			(ii) instante que a particula para momentaneamente:
			\begin{center}
				v(t) = -12t \\
				0 = -12t \\
				t = 0 s
			\end{center}			
		}
		\item (0,5 pt) Em que posição a partícula para (momentaneamente)?\\
		{ \color{blue}
			{\bf Resposta:} \\
			(i) posição que a particula para momentaneamente:
			\begin{center}
				x(t) = $4,0 - 6,0t^2$ \\
				x(0) = $4,0 - 6,0 \times 0^2$ \\
				x(0) = 4,0 m
			\end{center}			
		}
		\item (0,5 pt) Em que instante negativo e\\
		{ \color{blue}
			{\bf Resposta:} \\
			(i) instantes em que a partícula passa pela origem:
			\begin{center}
				x(t) = $4,0 - 6,0t^2$ \\
				0 = $4,0 - 6,0t^2$ \\
				$t^2 = 4/6$ \\
				t = $\pm \sqrt{\dfrac{4}{6}} = \pm \dfrac{2}{\sqrt{6}} = \pm \dfrac{2\sqrt{6}}{6} = \pm \dfrac{\sqrt{6}}{3} \cong \pm 0,82$ s
			\end{center}
			(ii) Logo, o instante negativo em que a partícula passa pela origem é aproximadamente -0,82 s.		
		}
		\item (0,5 pt) Em que instante positivo a partícula passa pela origem?\\
		{ \color{blue}
			{\bf Resposta:} \\
			(i) O instante positivo em que a partícula passa pela origem é aproximadamente 0,82 s (ver justificativa na letra (c)).		
		}
		\item (1,0 pt) Plote o gráfico de $x$ em função de $t$ \\
		para o intervalo de -5 s a + 5 s.\\
		{ \color{blue}
			{\bf Resposta:}}
			\begin{center}
				\begin{tikzpicture}
					\begin{axis}[
						xlabel=$t$,
						ylabel={$x$},
						xmin=-5,   xmax=5,
						ymin=-150,   ymax=20,
						axis lines=middle
					]
						% use TeX as calculator:
						\addplot{4 - 6*x^2};
					\end{axis}
				\end{tikzpicture}
			\end{center}
		\item (1,0 pt) Para deslocar a curva para a direita no gráfico,\\ devemos acrescentar a $x(t)$ o termo $+20t$ ou o termo $-20t$?\\
		{ \color{blue}
			{\bf Resposta:} Devemos acrescentar +20t. Logo x(t) ficaria: \\
			\begin{center}
				$x_{mod}(t) = 4,0 + 20t - 6,0t^2$
			\end{center}
		}
		\begin{center}
			\begin{tikzpicture}
			\begin{axis}[
			xlabel=$t$,
			ylabel={$x_{mod}$},
			xmin=-5,   xmax=5,
			ymin=-150,   ymax=50,
			axis lines=middle
			]
			% use TeX as calculator:
			\addplot{4 + 20*x -6*x^2};
			\end{axis}
			\end{tikzpicture}
		\end{center}
		\newpage
		
		\item (1,0 pt) Essa modificação aumenta ou diminui o valor de $x$ para o qual a partícula para momentaneamente?\\
		{ \color{blue}
			{\bf Resposta:} \\
			(i) obter v$_{mod}(t)$ a partir de x$_{mod}(t)$:
			\begin{center}
				v$_{mod}(t) =$ x'$_{mod}(t)$ $\therefore$\\
				v$_{mod}(t) = -12t + 20$
			\end{center}
			(ii) instante que a partícula para momentaneamente:
				\begin{eqnarray*}
					v_{mod}(t) 	& = & -12t + 20 \\
					0 			& = & -12t + 20 \\
					t 			& \cong & 1,67 s
				\end{eqnarray*}
			(ii) posição que a partícula para momentaneamente:
			\begin{eqnarray*}
				x_{mod}(t) 	& = & 4,0 + 20t - 6,0t^2 \\
				x_{mod}(1,67) 	& = & 4,0 + 20 \times 1,67 - 6,0 \times (1,67)^2 \\
				x_{mod}(1,67) 	& \cong & 4,0 + 33,4 -16,73 \\
				x_{mod}(1,67) 	& \cong & 20,67 m
			\end{eqnarray*}	
			A posição que partícula para é aproximadamente 20,67 m. Como antes a posição era 4,0 m, então a modificação aumenta o valor de x.
		}
	\end{enumerate}

	\item (5,0 pt) Em JavaScript, crie uma função {\tt posicao} que recebe {\tt t} como parâmetro (conforme equação apresentada na questão \ref{q:2-14}). A função deve retornar um número (a posição da partícula).
	
	{\color{blue} {\bf Resposta:} }
	
	\begin{lstlisting}[language=JavaScript]
	function posicao(t){
		var pos = 4 - 6*t*t;
		return pos;
	}\end{lstlisting}
	
	\end{enumerate}

\end{document}