\documentclass[12pt,a4paper,oneside]{article}

\usepackage[utf8]{inputenc}
\usepackage[portuguese]{babel}
\usepackage[T1]{fontenc}
\usepackage{amsmath}
\usepackage{amsfonts}
\usepackage{amssymb}
\usepackage{graphicx}

\usepackage{xcolor}
% Definindo novas cores
\definecolor{verde}{rgb}{0.25,0.5,0.35}
\definecolor{jpurple}{rgb}{0.5,0,0.35}
% Configurando layout para mostrar codigos Java
\usepackage{listings}
\lstset{
  language=Java,
  basicstyle=\ttfamily\small, 
  keywordstyle=\color{jpurple}\bfseries,
  stringstyle=\color{red},
  commentstyle=\color{verde},
  morecomment=[s][\color{blue}]{/**}{*/},
  extendedchars=true, 
  showspaces=false, 
  showstringspaces=false, 
  numbers=left,
  numberstyle=\tiny,
  breaklines=true, 
  backgroundcolor=\color{cyan!10}, 
  breakautoindent=true, 
  captionpos=b,
  xleftmargin=0pt,
  tabsize=4,
  escapeinside=||
}

\author{\\Universidade Federal de Goiás (UFG) - Regional Jataí\\Bacharelado em Ciência da Computação \\Física para Ciência da Computação \\Esdras Lins Bispo Jr.}

\title{\sc \huge Prova (Parte 1)}

\date{28 de março de 2017}

\begin{document}

\maketitle

{\bf ORIENTAÇÕES PARA A RESOLUÇÃO}

\footnotesize

\begin{itemize}
	\item A avaliação é individual, sem consulta;
	\item A pontuação máxima desta avaliação é 10,0 (dez) pontos, sendo uma das 05 (cinco) componentes que formarão a média final da disciplina: dois testes, duas provas e exercícios-bônus;
	\item A média final ($MF$) será calculada assim como se segue
	\begin{eqnarray}
		MF & = & MIN(10, S) \nonumber \\
		S & = & (\sum_{i=1}^{4} 0,2.T_i ) + 0,2.P  + EB \nonumber
	\end{eqnarray}
	em que 
	\begin{itemize}
		\item $S$ é o somatório da pontuação de todas as avaliações,
		\item $T_i$ é a pontuação obtida no teste $i$,
		\item $P$ é a pontuação obtida na prova, e
		\item $EB$ é a pontuação total dos exercícios-bônus.
	\end{itemize}
	\item O conteúdo exigido compreende os seguintes pontos apresentados no Plano de Ensino da disciplina: (1) Fundamentos Matemáticos, (2) Medidas Físicas e Vetores, e (3) Movimentos.
\end{itemize}


\begin{center}
	\fbox{\large Nome: \hspace{10cm}}
	\fbox{\large Assinatura: \hspace{9cm}}
\end{center}

\newpage

\normalsize

\begin{enumerate}
	
	\section*{Substitutiva do Teste 01}

	\item (5,0 pt) {\bf (Halliday 1.20)} O recorde para a maior garrafa de vidro foi estabelecido em 1992 por uma equipe de Millville, Nova Jersey, que criou uma garrafa com um volume de 193 galões americanos (admita que 1 galão americano seja 3785,41 cm$^3$, e a massa específica da água seja 1 g/cm$^3$).
		\begin{enumerate}
			\item Qual é a diferença entre esse volume e 1,0 milhão de centímetros cúbicos?
			\item Se a garrafa fosse enchida com água a uma vazão de 1,8 g/min, em quanto tempo estaria cheia? 
		\end{enumerate}
	
	
	\item (5,0 pt) Em JavaScript, crie um protótipo de objeto {\tt Calculadora} que tenha as propriedades de {\tt somar} e {\tt dividir}. Todas estas propriedades são operações binárias, recebem valores inteiros e retornam valores inteiros. Se, para as entradas fornecidas, não for possível gerar um valor de retorno válido, então exiba, via {\tt console.log}, o motivo do não retorno do valor.
	
	\section*{Substitutiva do Teste 02}
	
	\item (5,0 pt) {\bf (Halliday 2.14)} A função posição $x(t)$ de uma partícula que está se movendo ao longo do eixo $x$ é $x = 4,0 - 6,0t^2$, com $x$ em metros e $t$ em segundos. \label{q:2-14}
	\begin{enumerate}
		\item Em que instante e
		\item Em que posição a partícula para (momentaneamente)?
		\item Em que instante negativo e
		\item Em que instante positivo a partícula passa pela origem?
		\item Plote o gráfico de $x$ em função de $t$ \\
		para o intervalo de -5 s a + 5 s.
		\item Para deslocar a curva para a direita no gráfico,\\ devemos acrescentar a $x(t)$ o termo $+20t$ ou o termo $-20t$?
		\item Essa modificação aumenta ou diminui o valor de $x$ para o qual a partícula para momentaneamente?
	\end{enumerate}

	\item (5,0 pt) Em JavaScript, crie uma função {\tt posicao} que recebe {\tt t} como parâmetro (conforme equação apresentada na questão \ref{q:2-14}). A função deve retornar um número (a posição da partícula).
	
	\end{enumerate}

\end{document}