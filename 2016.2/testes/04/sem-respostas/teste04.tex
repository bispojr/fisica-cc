\documentclass[12pt,a4paper,oneside]{article}

\usepackage[utf8]{inputenc}
\usepackage[portuguese]{babel}
\usepackage[T1]{fontenc}
\usepackage{amsmath}
\usepackage{amsfonts}
\usepackage{amssymb}
\usepackage{graphicx}

\usepackage{xcolor}
% Definindo novas cores
\definecolor{verde}{rgb}{0.25,0.5,0.35}
\definecolor{jpurple}{rgb}{0.5,0,0.35}
% Configurando layout para mostrar codigos Java
\usepackage{listings}
\definecolor{mygreen}{rgb}{0,0.6,0}
\definecolor{mygray}{rgb}{0.5,0.5,0.5}
\definecolor{mymauve}{rgb}{0.58,0,0.82}

\usepackage{listings}
\lstdefinelanguage{JavaScript}{
	keywords={typeof, new, true, false, catch, function, return, null, catch, switch, var, if, in, while, do, else, case, break},
	keywordstyle=\color{black}\bfseries,
	ndkeywords={class, export, boolean, throw, implements, import, this},
	ndkeywordstyle=\color{darkgray}\bfseries,
	identifierstyle=\color{black},
	sensitive=false,
	comment=[l]{//},
	morecomment=[s]{/*}{*/},
	commentstyle=\color{black}\ttfamily,
	stringstyle=\color{black}\ttfamily,
	morestring=[b]',
	morestring=[b]",
}

\lstset{ %
	backgroundcolor=\color{white},   % choose the background color; you must add \usepackage{color} or \usepackage{xcolor}
	basicstyle=\small,        % the size of the fonts that are used for the code
	breakatwhitespace=false,         % sets if automatic breaks should only happen at whitespace
	breaklines=true,                 % sets automatic line breaking
	captionpos=b,                    % sets the caption-position to bottom
	commentstyle=\color{mygreen},    % comment style
	deletekeywords={...},            % if you want to delete keywords from the given language
	escapeinside={\%*}{*)},          % if you want to add LaTeX within your code
	extendedchars=true,              % lets you use non-ASCII characters; for 8-bits encodings only, does not work with UTF-8
	frame=single,	                   % adds a frame around the code
	keepspaces=true,                 % keeps spaces in text, useful for keeping indentation of code (possibly needs columns=flexible)
	keywordstyle=\color{blue},       % keyword style
	language=HTML,                 % the language of the code
	otherkeywords={*,...},           % if you want to add more keywords to the set
	numbers=left,                    % where to put the line-numbers; possible values are (none, left, right)
	numbersep=5pt,                   % how far the line-numbers are from the code
	numberstyle=\tiny\color{black}, % the style that is used for the line-numbers
	rulecolor=\color{black},         % if not set, the frame-color may be changed on line-breaks within not-black text (e.g. comments (green here))
	showspaces=false,                % show spaces everywhere adding particular underscores; it overrides 'showstringspaces'
	showstringspaces=false,          % underline spaces within strings only
	showtabs=false,                  % show tabs within strings adding particular underscores
	stepnumber=1,                    % the step between two line-numbers. If it's 1, each line will be numbered
	stringstyle=\color{mymauve},     % string literal style
	tabsize=2,	                   % sets default tabsize to 2 spaces
	title=\lstname,                   % show the filename of files included with \lstinputlisting; also try caption instead of title
	moredelim=**[is][\color{purple}]{@}{@},
}

\author{\\Universidade Federal de Goiás (UFG) - Regional Jataí\\Bacharelado em Ciência da Computação \\Física para Ciência da Computação \\Esdras Lins Bispo Jr.}

\title{\sc \huge Quarto Teste}

\date{07 de março de 2017}

\begin{document}

\maketitle

{\bf ORIENTAÇÕES PARA A RESOLUÇÃO}

\footnotesize

\begin{itemize}
	\item A avaliação é individual, sem consulta;
	\item A pontuação máxima desta avaliação é 10,0 (dez) pontos, sendo uma das 05 (cinco) componentes que formarão a média final da disciplina: dois testes, duas provas e exercícios-bônus;
	\item A média final ($MF$) será calculada assim como se segue
	\begin{eqnarray}
		MF & = & MIN(10, S) \nonumber \\
		S & = & (\sum_{i=1}^{4} 0,2.T_i ) + 0,2.P  + EB \nonumber
	\end{eqnarray}
	em que 
	\begin{itemize}
		\item $S$ é o somatório da pontuação de todas as avaliações,
		\item $T_i$ é a pontuação obtida no teste $i$,
		\item $P$ é a pontuação obtida na prova, e
		\item $EB$ é a pontuação total dos exercícios-bônus.
	\end{itemize}
	\item O conteúdo exigido compreende os seguintes pontos apresentados no Plano de Ensino da disciplina: (2) Medidas Físicas e Vetores, (3) Movimentos, (4) Trabalho e Energia, (5) Colisões, e (7) Outros Tópicos.
\end{itemize}


\begin{center}
	\fbox{\large Nome: \hspace{10cm}}
	\fbox{\large Assinatura: \hspace{9cm}}
\end{center}

\newpage

\normalsize

\begin{enumerate}

	\item (5,0 pt) {\bf (Halliday 4.21)} Um dardo é arremessado horizontalmente com uma velocidade inicial de 10 m/s em direção a um ponto P, o centro de um alvo de parede. O dardo atinge um ponto Q do alvo, verticalmente abaixo de P, 0,19 s depois do arremesso. \label{q:dardo}
	\begin{enumerate}
		\item Qual é a distância PQ?
		\item A que distância do alvo foi arremessado o dardo?
	\end{enumerate}	
	
	\item Em JavaScript, adapte as funções {\tt valoresIniciais} e {\tt emCadaPasso} conforme apresentada abaixo. Você deve substituir apenas as linhas 7, 10 e 14, pelos trechos de código 1, 2 e 3, respectivamente. O objetivo é que uma bola faça o movimento de queda livre de cabeça para baixo (ao invés da bola cair para baixo, ela deve ``cair para cima''). Lembre-se de que a bola quicará no teto até o repouso. Admita que em cada colisão com o teto, a bola transfere 20\% de sua energia cinética. Admita também que velocidade inicial da bola, no eixo y, seja zero e que $g = 9,8$ px/s$^2$.
	
	\begin{lstlisting}[language=JavaScript]
function valoresIniciais() {
	this.bola = new Bola(50, '#0000ff');
	
	this.bola.x = 100;
	this.bola.y = this.canvas.height - 50;
	
	//TRECHO 1
}
function emCadaPasso() {    
	if ( /* TRECHO 2 */ ) {
		this.bola.y += this.bola.vy;
		this.bola.vy += this.bola.ay;
	} 
	// TRECHO 3
	
	this.limparCanvas();
	this.bola.desenhar(this.contexto);
	this.bloco.desenhar(this.contexto);
}\end{lstlisting}
	
\end{enumerate}

\newpage

\section{Fórmulas Auxiliares}

\subsection{Movimento Uniforme (MU)}

\begin{enumerate}
	\item $x = x_0 + vt$ 
\end{enumerate}

\subsection{Movimento Uniformemente Variado (MUV)}

\begin{enumerate}
	\item $x - x_0 = v_0t + \frac{1}{2}at^2$ 
	\item $v = v_0 + at$ 
	\item $x - x_0 = \frac{1}{2}(v_0 + v)t$ 
	\item $x - x_0 = vt - \frac{1}{2} a t^2$
	\item $v^2 = v_0^2 + 2a(x - x_0)$
\end{enumerate}

\section*{Protótipo Bola}

\begin{lstlisting}[language=JavaScript]
function Bola(raio, cor) {
	this.raio = raio;
	this.cor = cor;
	this.x = 0;
	this.y = 0;
	this.vx = 0;
	this.vy = 0;
	this.ax = 0;
	this.ay = 0;
}
function desenhar(contexto){
	contexto.fillStyle = this.cor;
	contexto.beginPath();
	contexto.arc(this.x, this.y, this.raio, 0, 2 * Math.PI, true);
	contexto.closePath();
	contexto.fill();
}
function dentroLimiteInferior(canvas) {
	if (this.y + this.vy + this.raio <= canvas.height) {
		return true;
	} else {
		return false;
	}
}
function dentroLimiteSuperior(canvas) {
	if (this.y + this.vy + this.raio <= 0) {
		return true;
	} else {
		return false;
	}
}
function dentroLimiteDireito(canvas) {
	if (this.x + this.vx + this.raio <= canvas.width) {
		return true;
	} else {
		return false;
	}
}
Bola.prototype.desenhar = desenhar;
Bola.prototype.dentroLimiteInferior = dentroLimiteInferior;
Bola.prototype.dentroLimiteSuperior = dentroLimiteSuperior;
Bola.prototype.dentroLimiteDireito = dentroLimiteDireito;
\end{lstlisting}

\section*{Protótipo Bloco}

\begin{lstlisting}[language=JavaScript]
function Bloco(largura, altura, cor) {
	this.largura = largura;
	this.altura = altura;
	this.cor = cor;
	this.x = 0;
	this.y = 120;
}
function desenhar(contexto){    
	contexto.fillStyle = this.cor;
	contexto.beginPath();    
	contexto.rect(this.x, this.y, this.largura, this.altura);
	contexto.closePath();
	contexto.fill();
}
Bloco.prototype.desenhar = desenhar;
\end{lstlisting}

\section*{Protótipo Ambiente}

\begin{lstlisting}[language=JavaScript]
function Ambiente(){}
function iniciar() {    
	this.canvas = document.getElementById('canvas');
	this.contexto = this.canvas.getContext('2d');
	
	this.valoresIniciais();
	
	setInterval(this.emCadaPasso, 1000 / 60); // 60 fps
}
function valoresIniciais() {
	this.bola = new Bola(50, '#0000ff');
	
	this.bola.x = 210;
	this.bola.y = 70;
	
	this.bola.vx = 55 / 60;
	this.bola.vy = 0;
	
	this.bola.ax = 0;
	this.bola.ay = 9.8 / 60;
}
function emCadaPasso() {    
	if (this.bola.dentroLimiteInferior(this.canvas)) {
		this.bola.y += this.bola.vy;
		this.bola.vy += this.bola.ay;
	} 
	
	this.limparCanvas();
	this.bola.desenhar(this.contexto); // desenhe a bola 
	this.bloco.desenhar(this.contexto);
}
function limparCanvas(){
	this.contexto.clearRect(0, 0, this.canvas.width, this.canvas.height);
}
Ambiente.prototype.iniciar = iniciar;
Ambiente.prototype.valoresIniciais = valoresIniciais;
Ambiente.prototype.emCadaPasso = emCadaPasso;
Ambiente.prototype.limparCanvas = limparCanvas;
\end{lstlisting}

\section*{Arquivo Principal}

\begin{lstlisting}[language=JavaScript]
var amb = new Ambiente();
window.onload = amb.iniciar;
\end{lstlisting}

\end{document}