\documentclass[12pt,a4paper,oneside]{article}

\usepackage[utf8]{inputenc}
\usepackage[portuguese]{babel}
\usepackage[T1]{fontenc}
\usepackage{amsmath}
\usepackage{amsfonts}
\usepackage{amssymb}
\usepackage{graphicx}

\usepackage{xcolor}
% Definindo novas cores
\definecolor{verde}{rgb}{0.25,0.5,0.35}
\definecolor{jpurple}{rgb}{0.5,0,0.35}
% Configurando layout para mostrar codigos Java
\usepackage{listings}
\definecolor{mygreen}{rgb}{0,0.6,0}
\definecolor{mygray}{rgb}{0.5,0.5,0.5}
\definecolor{mymauve}{rgb}{0.58,0,0.82}

\usepackage{listings}
\lstdefinelanguage{JavaScript}{
	keywords={typeof, new, true, false, catch, function, return, null, catch, switch, var, if, in, while, do, else, case, break},
	keywordstyle=\color{blue}\bfseries,
	ndkeywords={class, export, boolean, throw, implements, import, this},
	ndkeywordstyle=\color{darkgray}\bfseries,
	identifierstyle=\color{black},
	sensitive=false,
	comment=[l]{//},
	morecomment=[s]{/*}{*/},
	commentstyle=\color{purple}\ttfamily,
	stringstyle=\color{red}\ttfamily,
	morestring=[b]',
	morestring=[b]",
}

\lstset{ %
	backgroundcolor=\color{white},   % choose the background color; you must add \usepackage{color} or \usepackage{xcolor}
	basicstyle=\small,        % the size of the fonts that are used for the code
	breakatwhitespace=false,         % sets if automatic breaks should only happen at whitespace
	breaklines=true,                 % sets automatic line breaking
	captionpos=b,                    % sets the caption-position to bottom
	commentstyle=\color{mygreen},    % comment style
	deletekeywords={...},            % if you want to delete keywords from the given language
	escapeinside={\%*}{*)},          % if you want to add LaTeX within your code
	extendedchars=true,              % lets you use non-ASCII characters; for 8-bits encodings only, does not work with UTF-8
	frame=single,	                   % adds a frame around the code
	keepspaces=true,                 % keeps spaces in text, useful for keeping indentation of code (possibly needs columns=flexible)
	keywordstyle=\color{blue},       % keyword style
	language=HTML,                 % the language of the code
	otherkeywords={*,...},           % if you want to add more keywords to the set
	numbers=left,                    % where to put the line-numbers; possible values are (none, left, right)
	numbersep=5pt,                   % how far the line-numbers are from the code
	numberstyle=\tiny\color{mygray}, % the style that is used for the line-numbers
	rulecolor=\color{black},         % if not set, the frame-color may be changed on line-breaks within not-black text (e.g. comments (green here))
	showspaces=false,                % show spaces everywhere adding particular underscores; it overrides 'showstringspaces'
	showstringspaces=false,          % underline spaces within strings only
	showtabs=false,                  % show tabs within strings adding particular underscores
	stepnumber=1,                    % the step between two line-numbers. If it's 1, each line will be numbered
	stringstyle=\color{mymauve},     % string literal style
	tabsize=2,	                   % sets default tabsize to 2 spaces
	title=\lstname,                   % show the filename of files included with \lstinputlisting; also try caption instead of title
	moredelim=**[is][\color{purple}]{@}{@},
}

\author{\\Universidade Federal de Goiás (UFG) - Regional Jataí\\Bacharelado em Ciência da Computação \\Física para Ciência da Computação \\Esdras Lins Bispo Jr.}

\title{\sc \huge Terceiro Teste}

\date{14 de fevereiro de 2017}

\begin{document}

\maketitle

{\bf ORIENTAÇÕES PARA A RESOLUÇÃO}

\footnotesize

\begin{itemize}
	\item A avaliação é individual, sem consulta;
	\item A pontuação máxima desta avaliação é 10,0 (dez) pontos, sendo uma das 05 (cinco) componentes que formarão a média final da disciplina: dois testes, duas provas e exercícios-bônus;
	\item A média final ($MF$) será calculada assim como se segue
	\begin{eqnarray}
		MF & = & MIN(10, S) \nonumber \\
		S & = & (\sum_{i=1}^{4} 0,2.T_i ) + 0,2.P  + EB \nonumber
	\end{eqnarray}
	em que 
	\begin{itemize}
		\item $S$ é o somatório da pontuação de todas as avaliações,
		\item $T_i$ é a pontuação obtida no teste $i$,
		\item $P$ é a pontuação obtida na prova, e
		\item $EB$ é a pontuação total dos exercícios-bônus.
	\end{itemize}
	\item O conteúdo exigido compreende os seguintes pontos apresentados no Plano de Ensino da disciplina: (2) Medidas Físicas e Vetores, (3) Movimentos, e (5) Colisões.
\end{itemize}


\begin{center}
	\fbox{\large Nome: \hspace{10cm}}
	\fbox{\large Assinatura: \hspace{9cm}}
\end{center}

\newpage

\normalsize

\begin{enumerate}

	\item (5,0 pt) {\bf (Halliday 2.60)} Uma pedra é lançada verticalmente para cima a partir do solo no instante $t = 0$ s. Em  $t  = 1,5$ s, a pedra ultrapassa o alto de uma torre; $1,0$ s depois, atinge a altura máxima. Qual é a altura da torre?
	
	{\color{blue} {\bf Resposta:} \\
		As equações abaixo estão considerando o movimento de descida da pedra, i.e., após a pedra atingir a altura máxima e começar a se mover no sentido oposto ao eixo $y$.\\
		\\
		Velocidade da pedra no instante que ultrapassa o alto da torre $\therefore$ \\
		$v = v_0 + at$\\
		$v = 0 -9,8$ m/s$^2 \times 1$ s\\
		$v = -9,8$ m/s\\
		\\
		Altura da torre $\therefore$\\
		$y - y_0 = v_0t + \frac{1}{2} a t^2$ \\
		$\Delta y = -9,8$ m/s$^2 \times 1,5$ s $+\frac{1}{2} (-9,8$ m/s$^2)(1,5$ s$)^2$\\
		$\Delta y = -14,7$ m $-11,025$ m\\
		$\Delta y = -25,725$ m\\
		\\
		$\Delta y$ assumiu valor negativo, pois a pedra está caindo (movendo-se no sentido negativo do eixo $y$). Logo, a altura da torre é 25,725 m. 
	}
	
	\newpage
	
	\item Em JavaScript, reescreva a função {\tt emCadaPassoX}, conforme vista em sala de aula, de forma que a bola azul ao chegar no limite direito do {\tt canvas}, ela volte na mesma direção, i.e., ela fará um movimento uniforme (MU) desta vez com a velocidade negativa. Garanta que os MUs de ida e volta permaneçam indefinidamente.
	
	{\color{blue} \bf Resposta: }
	
	\begin{lstlisting}[language=JavaScript]
function emCadaPassoX(){
	if(
		(bola.x + bola.raio + bola.vx < canvas.width) &&
		(bola.x - bola.raio + bola.vx > 0)
	){
		bola.x += bola.vx;
	}
	else{
		bola.vx = -bola.vx;
	}
	
	bola.desenhar(contexto);
}\end{lstlisting}
	
\end{enumerate}

\section{Fórmulas Auxiliares}

\subsection{Movimento Uniforme (MU)}

\begin{enumerate}
	\item $x = x_0 + vt$ 
\end{enumerate}

\subsection{Movimento Uniformemente Variado (MUV)}

\begin{enumerate}
	\item $x - x_0 = v_0t + \frac{1}{2}at^2$ 
	\item $v = v_0 + at$ 
	\item $x - x_0 = \frac{1}{2}(v_0 + v)t$ 
	\item $x - x_0 = vt - \frac{1}{2} a t^2$
	\item $v^2 = v_0^2 + 2a(x - x_0)$
\end{enumerate}

\end{document}