\documentclass[12pt,a4paper,oneside]{article}

\usepackage[utf8]{inputenc}
\usepackage[portuguese]{babel}
\usepackage[T1]{fontenc}
\usepackage{amsmath}
\usepackage{amsfonts}
\usepackage{amssymb}
\usepackage{graphicx}

\usepackage{xcolor}
% Definindo novas cores
\definecolor{verde}{rgb}{0.25,0.5,0.35}
\definecolor{jpurple}{rgb}{0.5,0,0.35}
% Configurando layout para mostrar codigos Java
\usepackage{listings}
\lstset{
  language=Java,
  basicstyle=\ttfamily\small, 
  keywordstyle=\color{jpurple}\bfseries,
  stringstyle=\color{red},
  commentstyle=\color{verde},
  morecomment=[s][\color{blue}]{/**}{*/},
  extendedchars=true, 
  showspaces=false, 
  showstringspaces=false, 
  numbers=left,
  numberstyle=\tiny,
  breaklines=true, 
  backgroundcolor=\color{cyan!10}, 
  breakautoindent=true, 
  captionpos=b,
  xleftmargin=0pt,
  tabsize=4,
  escapeinside=||
}

\author{\\Universidade Federal de Goiás (UFG) - Regional Jataí\\Bacharelado em Ciência da Computação \\Física para Ciência da Computação \\Esdras Lins Bispo Jr.}

\title{\sc \huge Terceiro Teste}

\date{14 de fevereiro de 2017}

\begin{document}

\maketitle

{\bf ORIENTAÇÕES PARA A RESOLUÇÃO}

\footnotesize

\begin{itemize}
	\item A avaliação é individual, sem consulta;
	\item A pontuação máxima desta avaliação é 10,0 (dez) pontos, sendo uma das 05 (cinco) componentes que formarão a média final da disciplina: dois testes, duas provas e exercícios-bônus;
	\item A média final ($MF$) será calculada assim como se segue
	\begin{eqnarray}
		MF & = & MIN(10, S) \nonumber \\
		S & = & (\sum_{i=1}^{4} 0,2.T_i ) + 0,2.P  + EB \nonumber
	\end{eqnarray}
	em que 
	\begin{itemize}
		\item $S$ é o somatório da pontuação de todas as avaliações,
		\item $T_i$ é a pontuação obtida no teste $i$,
		\item $P$ é a pontuação obtida na prova, e
		\item $EB$ é a pontuação total dos exercícios-bônus.
	\end{itemize}
	\item O conteúdo exigido compreende os seguintes pontos apresentados no Plano de Ensino da disciplina: (2) Medidas Físicas e Vetores, (3) Movimentos, e (5) Colisões.
\end{itemize}


\begin{center}
	\fbox{\large Nome: \hspace{10cm}}
	\fbox{\large Assinatura: \hspace{9cm}}
\end{center}

\newpage

\normalsize

\begin{enumerate}

	\item (5,0 pt) {\bf (Halliday 2.60)} Uma pedra é lançada verticalmente para cima a partir do solo no instante $t = 0$ s. Em  $t  = 1,5$ s, a pedra ultrapassa o alto de uma torre; $1,0$ s depois, atinge a altura máxima. Qual é a altura da torre?
	
	\item Em JavaScript, reescreva a função {\tt emCadaPassoX}, conforme vista em sala de aula, de forma que a bola azul ao chegar no limite direito do {\tt canvas}, ela volte na mesma direção, i.e., ela fará um movimento uniforme (MU) desta vez com a velocidade negativa. Garanta que os MUs de ida e volta permaneçam indefinidamente.
	
\end{enumerate}

\section{Fórmulas Auxiliares}

\subsection{Movimento Uniforme (MU)}

\begin{enumerate}
	\item $x = x_0 + vt$ 
\end{enumerate}

\subsection{Movimento Uniformemente Variado (MUV)}

\begin{enumerate}
	\item $x - x_0 = v_0t + \frac{1}{2}at^2$ 
	\item $v = v_0 + at$ 
	\item $x - x_0 = \frac{1}{2}(v_0 + v)t$ 
	\item $x - x_0 = vt - \frac{1}{2} a t^2$
	\item $v^2 = v_0^2 + 2a(x - x_0)$
\end{enumerate}

\end{document}