\documentclass[12pt,a4paper,oneside]{article}

\usepackage[utf8]{inputenc}
\usepackage[portuguese]{babel}
\usepackage[T1]{fontenc}
\usepackage{amsmath}
\usepackage{amsfonts}
\usepackage{amssymb}
\usepackage{graphicx}

\usepackage{xcolor}
% Definindo novas cores
\definecolor{verde}{rgb}{0.25,0.5,0.35}
\definecolor{jpurple}{rgb}{0.5,0,0.35}
% Configurando layout para mostrar codigos Java
\usepackage{listings}
\lstset{
  language=Java,
  basicstyle=\ttfamily\small, 
  keywordstyle=\color{jpurple}\bfseries,
  stringstyle=\color{red},
  commentstyle=\color{verde},
  morecomment=[s][\color{blue}]{/**}{*/},
  extendedchars=true, 
  showspaces=false, 
  showstringspaces=false, 
  numbers=left,
  numberstyle=\tiny,
  breaklines=true, 
  backgroundcolor=\color{cyan!10}, 
  breakautoindent=true, 
  captionpos=b,
  xleftmargin=0pt,
  tabsize=4,
  escapeinside=||
}

\author{\\Universidade Federal de Goiás (UFG) - Regional Jataí\\Bacharelado em Ciência da Computação \\Física para Ciência da Computação \\Esdras Lins Bispo Jr.}

\title{\sc \huge Primeiro Teste}

\date{02 de setembro de 2019}

\begin{document}

\maketitle

{\bf ORIENTAÇÕES PARA A RESOLUÇÃO}

\footnotesize

\begin{itemize}
	\item A avaliação é individual, sem consulta;
	\item A pontuação máxima desta avaliação é 10,0 (dez) pontos, sendo uma das 06 (seis) componentes que formarão a média final da disciplina: quatro mini-testes (MT), uma prova final (PF) e eventuais exercícios-bônus propostos (EB);
	\item A média final ($MF$) será calculada assim como se segue
	\begin{eqnarray}
	MF & = & MIN(10, S) \nonumber \\
	S & = & [(\sum_{i=1}^{4} max(MT_i, SMT_i ) + PF].0,2  + EB \nonumber
	\end{eqnarray}
	em que 
	\begin{itemize}
		\item $S$ é o somatório da pontuação de todas as avaliações, e
		\item $SMT_i$ é a substitutiva do mini-teste $i$.
	\end{itemize}
	\item O conteúdo exigido compreende os seguintes pontos apresentados no Plano de Ensino da disciplina: (1) Medição e Grandezas Físicas, e (2) Tecnologias Básicas.
\end{itemize}


\begin{center}
	\fbox{\large Nome: \hspace{10cm}}
	\fbox{\large Assinatura: \hspace{9cm}}
\end{center}

\newpage

\normalsize

\begin{enumerate}

	\item (5,0 pt) {\bf (Halliday 1.21  [Adaptado])} 
	\begin{enumerate}
		\item Supondo que a água tenha uma massa específica de exatamente 1 g/$\mbox{cm}^3$, determine a massa de 5 metro cúbico de água em quilogramas.
		\item Suponha que sejam necessárias 7,0 h para drenar um recipiente com 4900 $m^3$ de água. Qual é a ``vazão mássica'' da água do recipiente, em quilogramas por segundo?
	\end{enumerate}
	
	\item (5,0 pt) Em JavaScript, crie um protótipo de objeto {\tt Planeta} que tenha as propriedades (i) {\tt nome}, (ii) {\tt raio}, (iii) {\tt distanciaTerra}, e (iv) {\tt descricao}. O {\tt nome} é uma cadeia; a {\tt raio} e a {\tt distanciaTerra} são valores numéricos (em quilômetros); e a {\tt descricao} é uma função que exibe, via {\tt console.log}, todas as demais propriedades de {\tt Planeta}. Crie um objeto a partir de {\tt Planeta}. Atribua valores para as proprieda\-des ao seu gosto.
	
	\end{enumerate}
\end{document}