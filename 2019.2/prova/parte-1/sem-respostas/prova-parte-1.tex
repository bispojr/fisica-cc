\documentclass[12pt,a4paper,oneside]{article}

\usepackage[utf8]{inputenc}
\usepackage[portuguese]{babel}
\usepackage[T1]{fontenc}
\usepackage{amsmath}
\usepackage{amsfonts}
\usepackage{amssymb}
\usepackage{graphicx}

\usepackage{xcolor}
% Definindo novas cores
\definecolor{verde}{rgb}{0.25,0.5,0.35}
\definecolor{jpurple}{rgb}{0.5,0,0.35}
% Configurando layout para mostrar codigos Java
\usepackage{listings}
\lstset{
  language=Java,
  basicstyle=\ttfamily\small, 
  keywordstyle=\color{jpurple}\bfseries,
  stringstyle=\color{red},
  commentstyle=\color{verde},
  morecomment=[s][\color{blue}]{/**}{*/},
  extendedchars=true, 
  showspaces=false, 
  showstringspaces=false, 
  numbers=left,
  numberstyle=\tiny,
  breaklines=true, 
  backgroundcolor=\color{cyan!10}, 
  breakautoindent=true, 
  captionpos=b,
  xleftmargin=0pt,
  tabsize=4,
  escapeinside=||
}

\author{\\Universidade Federal de Goiás (UFG) - Regional Jataí\\Bacharelado em Ciência da Computação \\Física para Ciência da Computação \\Esdras Lins Bispo Jr.}

\title{\sc \huge Prova (Parte 1)}

\date{25 de novembro de 2019}

\begin{document}

\maketitle

{\bf ORIENTAÇÕES PARA A RESOLUÇÃO}

\footnotesize

\begin{itemize}
	\item A avaliação é individual, sem consulta;
	\item A pontuação máxima desta avaliação é 10,0 (dez) pontos, sendo uma das 05 (cinco) componentes que formarão a média final da disciplina: dois testes, duas provas e exercícios-bônus;
	\item A média final ($MF$) será calculada assim como se segue
	\begin{eqnarray}
		MF & = & MIN(10, S) \nonumber \\
		S & = & (\sum_{i=1}^{4} 0,2.T_i ) + 0,2.P  + EB \nonumber
	\end{eqnarray}
	em que 
	\begin{itemize}
		\item $S$ é o somatório da pontuação de todas as avaliações,
		\item $T_i$ é a pontuação obtida no teste $i$,
		\item $P$ é a pontuação obtida na prova, e
		\item $EB$ é a pontuação total dos exercícios-bônus.
	\end{itemize}
	\item O conteúdo exigido compreende os seguintes pontos apresentados no Plano de Ensino da disciplina: (1) Medição e Grandezas Físicas, (2) Movimentos, e (6) Tecnologias Básicas.
\end{itemize}


\begin{center}
	\fbox{\large Nome: \hspace{10cm}}
	\fbox{\large Assinatura: \hspace{9cm}}
\end{center}

\newpage

\normalsize

\begin{enumerate}
	
	\section*{Substitutiva do Teste 01}

\item (5,0 pt) {\bf (Halliday 1.21 Adaptada)} 
\begin{enumerate}
	\item Supondo que a água tenha uma massa específica de exatamente 1 g/$\mbox{cm}^3$, determine a massa de dez metros cúbicos de água em quilogramas.
	\item Suponha que sejam necessárias 10,0 h para drenar um recipiente com 7200 $m^3$ de água. Qual é a ``vazão mássica'' da água do recipiente, em quilogramas por segundo?
\end{enumerate}		 
	
	
	\item (5,0 pt) Em JavaScript, crie um protótipo de objeto {\tt Calculadora} que tenha as propriedades de {\tt subtrair} e {\tt multiplicar}. Todas estas propriedades são operações binárias, recebem valores inteiros e retornam valores inteiros. Se, para as entradas fornecidas, não for possível gerar um valor de retorno válido, então exiba, via {\tt console.log}, o motivo do não retorno do valor.
	
	\section*{Substitutiva do Teste 02}
	
	\item (5,0 pt) {\bf (Halliday 2.15 Adaptada)} Se a posição de uma partícula é dada por $x = 15 -8t  + t^2$ (onde $t$ está em segundos e $x$ em metros): \label{questao}
	\begin{enumerate}
		\item Qual é a velocidade da partícula em $t = 1$s? 
		\item O movimento nesse instante é no sentido positivo ou negativo de $x$? 
		\item Qual é a velocidade escalar da partícula nesse instante?
		\item A velocidade escalar está aumentando	ou diminuindo nesse instante?
		\item Existe algum instante no 	qual a velocidade se anula? Caso a resposta seja afirmativa, para que valor de $t$ isso acontece? 
		\item Existe algum instante após  $t  = 5$s	no qual a partícula está se movendo no sentido negativo de $x$?  Caso a resposta seja afirmativa, para que valor de $t$ isso acontece?
	\end{enumerate}

	\item (5,0 pt) Em JavaScript, crie uma função {\tt posicao} que recebe {\tt t} como parâmetro (conforme equação apresentada na questão \ref{questao}). A função deve retornar um número (a posição da partícula).
	
	\end{enumerate}

\end{document}